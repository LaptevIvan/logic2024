\documentclass[aspectratio=169]{beamer}
\usepackage[utf8]{inputenc}
\usepackage[english,russian]{babel}
\usepackage{cancel}
\usepackage{amssymb}
\usepackage{stmaryrd}
\usepackage{cmll}
\usepackage{graphicx}
\usepackage{amsthm}
\usepackage{tikz}
\usepackage{multicol}
\usetikzlibrary{patterns}
\usepackage{chronosys}
\usepackage{proof}
\usepackage{multirow}
\setbeamertemplate{navigation symbols}{}
%\usetheme{Warsaw}

\newtheorem{thm}{Теорема}[section]
\newtheorem{dfn}{Определение}[section]
\newtheorem{lmm}{Лемма}[section]
\newtheorem{exm}{Пример}[section]
\newtheorem{snote}{Пояснение}[section]

\newcommand{\divisible}%                                                     
{\mathrel{\lower.2ex%
\vbox{\baselineskip=0.7ex\lineskiplimit=0pt%
\kern6pt \hbox{.}\hbox{.}\hbox{.}}%
}}

\begin{document}

\newcommand\doubleplus{+\kern-1.3ex+\kern0.8ex}
\newcommand\mdoubleplus{\ensuremath{\mathbin{+\mkern-10mu+}}}

\begin{frame}{}
\LARGE\begin{center}Трансфинитная индукция\end{center}
\end{frame}

\begin{frame}{Два вида индукции}
\begin{dfn}[принцип математической индукции]
Какое бы ни было $\varphi(x)$, если $\varphi(0)$ и при всех $x$ выполнено $\varphi(x)\rightarrow \varphi(x')$, то
при всех $x$ выполнено и само $\varphi(x)$.
\end{dfn}

\begin{dfn}[принцип полной математической индукции]
Какое бы ни было $\psi(x)$, если $\psi(0)$ и при всех $x$ выполнено $(\forall t.t \leq x \rightarrow \psi(t))\rightarrow \psi(x')$, то
при всех $x$ выполнено и само $\psi(x)$.
\end{dfn}

\begin{thm}Принципы математической индукции эквивалентны\end{thm}
\begin{proof}
$(\Rightarrow)$ взяв $\varphi := \psi$, имеем выполненность $\varphi(x)\rightarrow\varphi(x')$, значит, $\forall x.\psi(x)$. \pause\\
$(\Leftarrow)$ возьмём $\psi(x) := \forall t.t\le x\rightarrow\varphi(t)$.
\end{proof}
\end{frame}

\begin{frame}{Наследственные подмножества}
\begin{dfn} Назовём вполне упорядоченное отношением $(\in)$ множество $S$ наследственным подмножеством $A$, если 
$\forall x.x \in A \rightarrow (\forall t.t \in x \rightarrow t \in S) \rightarrow x \in S$.
\end{dfn}
\begin{thm}Единственным наследственным подмножеством вполне упорядоченного множества является оно само.\end{thm}
\begin{proof}Пусть $B \subseteq A$ --- наследственное и $B \ne A$.
Тогда существует $a = \min (A \setminus B)$. Тогда $(\forall t.t \in a \rightarrow t \in B) \rightarrow a \in B$ по наследственности $B$,
и выполнено $\forall t.t \in a \rightarrow t \in B$ (по минимальности $a$). Значит, $a \in B$.
\end{proof}
\end{frame}

\begin{frame}{Трансфинитная индукция}
\begin{thm}[ограниченная трансфинитная индукция] Если для $\varphi(x)$ (некоторого утверждения
теории множеств) и некоторого ординала $\varepsilon$ (ограничения) выполнено
$\forall x.x \in \varepsilon \rightarrow (\forall t.t \in x \rightarrow \varphi(t)) \rightarrow \varphi(x)$,
то $\forall x.x \in \varepsilon \rightarrow \varphi(x)$.
\end{thm}
\begin{proof}Рассмотрим $S = \{ x\in \varepsilon\ |\ \varphi(x) \}$. Тогда $x \in S$ равносильно 
$x\in\varepsilon\with\varphi(x)$.
Тогда перепишем: $\forall e.e \in \varepsilon \rightarrow (\forall x.x \in e \rightarrow x \in S) \rightarrow e \in S$.
Отсюда по теореме о наследственных множествах $S = \varepsilon$.\end{proof}

\begin{thm}[неограниченная трансфинитная индукция] Если для $\varphi(x)$ (некоторого утверждения
теории множеств) выполнено
$\forall x.\text{ординал}(x) \rightarrow (\forall t.t \in x \rightarrow \varphi(t)) \rightarrow \varphi(x)$,
то $\forall x.\text{ординал}(x) \rightarrow \varphi(x)$.
\end{thm}


\end{frame}

\begin{frame}{Альтернативная формулировка}
\begin{thm}Для ординала $\varepsilon$ подмножество $S \in \varepsilon$ --- наследственное, % тогда и только тогда, когда 
если и только если одновременно:\\
Если $x \in \varepsilon$ и $x = \varnothing$, то $x \in S$;\\
Если $x \in \varepsilon$ и существует $y$: $y' = x$, то $y \in S \rightarrow x \in S$;\\
Если $x \in \varepsilon$ и $x$ --- предельный, то $(\forall t.t \in x \rightarrow t \in S) \rightarrow (x \in S)$.
\end{thm}

\begin{proof}$(\Rightarrow)$ очевидно. \pause Докажем $(\Leftarrow)$: пусть $S$ не наследственное: 
$E := \{e \in \varepsilon \ |\  (\forall t.t \in e \rightarrow t \in S) \with e \notin S \}$
и $E \ne \varnothing$. Тогда пусть $e = \min E$. %Возможны три варианта:

\begin{enumerate}
\item $e = \varnothing$ или предельный. Тогда $(\forall t.t \in e \rightarrow t \in S) \rightarrow (e \in S)$.
\item $e = y'$. Тогда $y \in \varepsilon$ ($\varepsilon$ --- ординал) и 
$(\forall t.t \in y \rightarrow t \in S) \rightarrow (y \in S)$ (так как $e$ минимальный, для которого $S$ не наследственное). \pause
По условию, $(y \in S) \rightarrow (e \in S)$, отсюда $(\forall t.t \in e \rightarrow t \in S) \rightarrow (e \in S)$.

\begin{center}\tikz{\draw[thick,-stealth] (0,0) -- (7,0); 
\filldraw (2,0) circle (1pt);
\filldraw (1,0) circle (1pt);
\filldraw (3,0) node[above] {$t \in y$} circle (2pt);
\filldraw[red] (4,0) node[above] {$y$} circle (2pt) ;
\filldraw (5,0) node[above] {$e\vphantom{y}$} circle (2pt) ; }\end{center}

\end{enumerate}\vspace{-0.3cm}
\end{proof}

%\begin{thm}[Принцип трансфинитной индукции] Если для $\varphi(x)$ и некоторого ординала $\varepsilon$ выполнено:
%$\varphi(\varnothing) \with \varphi(x) \rightarrow \varphi(x') \with asdf 
%\end{thm}
\end{frame}

\begin{frame}{Пример применения: $\alpha\cdot\alpha = \alpha$ при $\alpha \ge \aleph_0$}
\begin{thm}Если $\alpha$ --- кардинальное число, $\alpha \ge \aleph_0$, 
то $\alpha\cdot\alpha = \alpha$.\end{thm}
\begin{proof}Трансфинитная индукция: $\varphi(x) := x < \omega \vee x \cdot x = x$
\begin{enumerate}
\item База: $x = \varnothing$. Тогда 
$\varphi(\varnothing) \equiv \varnothing < \omega \vee |\varnothing \times \varnothing| = \varnothing$,
что доказуемо.
\item Переход: $\forall y.y < x \rightarrow \varphi(y)$, тогда $\varphi(x)$. Три случая:
\begin{enumerate}
\item $x < \omega$. Тогда $\varphi(x)$ истинно (аналогично базе).
\item $x = \omega$. Счётный случай (рассмотрим отдельно).
\item $x > \omega$. Общий случай (рассмотрим отдельно).
\end{enumerate}
\end{enumerate}
\end{proof}
\end{frame}

\begin{frame}{Счётный случай: $\omega < \omega \vee |\omega \cdot \omega| = \omega$}
Тогда $\omega \times \omega$ упорядочим так: $\langle p,q \rangle \prec \langle s,t \rangle$,
если \begin{enumerate}
\item $\max(p,q) < \max(s,t)$
\item $\max(p,q) = \max(s,t)$ и $q < t$
\item $\max(p,q) = \max(s,t)$, $q = t$ и $p < s$
\end{enumerate}
Очевидно, можно построить биекцию между так упорядоченными значениями и $\omega$.

\begin{center}\begin{tikzpicture}

\filldraw[gray!20] (0,0) -- (4.5, 0) -- (4.5, 3) -- (0, 3) -- cycle;
\filldraw[gray!50] (0,0) -- (3, 0) -- (3, 2) -- (0, 2) -- cycle;
\filldraw[gray] (0,0) -- (1.5, 0) -- (1.5, 1) -- (0, 1) -- cycle;

\foreach \x in {0, 1, 2, 3} {
	\foreach \y in {0, 1, 2, 3} {
                \node at (1.5*\x + 1, \y + 0.6) {
			\pgfmathparse{(max(\x,\y))*(max(\x,\y)) + \y + ((\y+1)==(max(\x,\y)+1))*\x}%
			\pgfmathprintnumber{\pgfmathresult}%
		};
		\node at (1.5*\x + 0.4, \y + 0.2) {\footnotesize $\langle \x,\y \rangle$};
		\draw (1.5*\x, \y) -- (1.5*\x +1.5, \y) -- (1.5*\x +1.5, \y +1) -- (1.5*\x, \y +1) -- cycle;
	}
}

\end{tikzpicture}\end{center}
\end{frame}

\begin{frame}{Общий случай: $|\alpha \cdot \alpha| = \alpha$}
Аналогично счётному случаю, $\alpha \times \alpha$ упорядочим так: $\langle p,q \rangle \prec \langle s,t \rangle$,
если \begin{enumerate}
\item $p \cup q < s \cup t$
\item $p \cup q = s \cup t$ и $q < t$
\item $p \cup q = s \cup t$, $q = t$ и $p < s$
\end{enumerate}
\begin{itemize}
\item Легко заметить, что это --- линейный порядок (показав, что $p \not\prec q$ и $q \not\prec p$ влечёт $p = q$)
\item ... и полный порядок. Найти наименьший в $S \ne \varnothing$ возможно, рассмотрев $m_1 := \min \{ p \cup q\ |\ \langle p,q \rangle \in S\}$ и
$M_1 := \{ \langle p,q\rangle\ |\ \langle p,q \rangle \in S, p \cup q = m_1\}$,
затем $m_2 := \min \{q\ |\ \langle p,q \rangle \in M_1 \}$,
$M_2 := \{\langle p,q\rangle\ |\ \langle p,q \rangle \in M_1, q = m_1\}$.
Тогда требуемым наименьшим в $S$ будет $\min \{ p\ |\ \langle p,q \rangle \in M_2\}$
\item Тогда $\langle \alpha\times\alpha, (\prec)\rangle$ соответствует какой-то ординал $\tau$ 
и сохраняющая порядок биекция $t: \tau\rightarrow\alpha\times\alpha$. 
\item Заметим, что $x < \omega$ тогда и только тогда, когда $\cup(\cup t(x)) < \omega$
(очевидно из того, что $|\{z\ |\ \text{ординал}(z), z < x\}|=|\{p\ |\ p \prec t(x)\}|$).
\item Покажем, что $|\tau| = \alpha$.
\end{itemize}
\end{frame}

\begin{frame}{Докажем $\tau = \alpha$}

{\color{gray}
$\langle \alpha\times\alpha, (\prec)\rangle$ соответствует какой-то ординал $\tau$\\$t: \tau\rightarrow\alpha\times\alpha$
}\vspace{0.3cm}

Очевидно, что $\tau \ge \alpha$ (так как $|\tau| = |\alpha\times\alpha| \ge \alpha$). Но пусть $\tau > \alpha$.
\begin{itemize}
\item Тогда $t(\alpha) = \langle\zeta,\eta\rangle$ определено (у $\alpha$ есть образ).
\item Пусть $\sigma := \zeta \cup \eta$. Очевидно, $\langle \zeta, \eta \rangle \preceq \langle \sigma,\sigma \rangle$
и $\sigma \in \alpha$.
%При этом $\omega \le \sigma$, так как иначе $\zeta < \omega$ и $\eta < \omega$, и поэтому $\alpha < \omega$.
\item Каков образ $t$ на этом начальном отрезке?
$\{t(x)\ |\ x < \alpha\} \subseteq \{\langle p,q\rangle\ |\ p,q \le \sigma\}$.
Поэтому $\alpha \le |(\sigma+1)\times(\sigma+1)|$. 
\item С другой стороны, $\sigma < \alpha$. Поскольку $\alpha$ --- кардинал (т.е., в частности, предельный ординал), 
то $\sigma+1 < \alpha$ и $|\sigma+1| < \alpha$. 
\item По предположению индукции, $|\sigma+1|<\omega \vee |\sigma+1| = |\sigma+1|\cdot|\sigma+1|$,
по свойствам $(\prec)$ имеем $\sigma\ge\omega$.
\item Отсюда $\alpha \le |(\sigma+1)\times(\sigma+1)| = |\sigma+1| < \alpha$, что невозможно.
\end{itemize}

%Предположим, что $|\beta| = |\alpha\times\alpha| > \alpha$. Пусть $f(\alpha) = \langle \zeta,\eta \rangle$
%и $\gamma = \zeta \cup \beta$. Тогда $\langle \zeta, \eta \rangle \preceq \langle \gamma,\gamma\rangle$.

\end{frame}

\begin{frame}{}
\LARGE\begin{center}Теорема о непротиворечивости формальной арифметики\end{center}
\end{frame}

\begin{frame}{Исчисление $S_\infty$}
\begin{enumerate}
\item Язык: связки $\neg$, $\vee$, $\forall$, $=$; нелогические символы: $(+)$,$(\cdot)$,$(')$,$0$; переменные: $x$.
\item Аксиомы: все истинные формулы вида $\theta_1=\theta_2$; все истинные отрицания формул вида $\neg\theta_1=\theta_2$
($\theta_i$ --- термы без переменных).
\item Структурные (слабые) правила:
$$\infer{\zeta\vee\beta\vee\alpha\vee\delta}{\zeta\vee\alpha\vee\beta\vee\delta} \quad\quad
\infer{\alpha\vee\delta}{\alpha\vee\alpha\vee\delta}$$

сильные правила
$$\infer{\alpha\vee\beta}{\beta}\quad
\infer{\neg(\alpha\vee\beta)\vee\delta}{\neg\alpha\vee\delta\quad\neg\beta\vee\delta}\quad
\infer{\neg\neg\alpha\vee\delta}{\alpha\vee\delta}\quad
\infer{(\neg\forall x.\alpha)\vee\delta}{\neg\alpha[x := \theta]\vee\delta}\quad$$

Формулы в правилах, обозначенные буквами $\zeta$ и $\delta$, называются боковыми и могут отсутствовать.
\item и ещё два правила \dots
\end{enumerate}
\end{frame}

\begin{frame}{Ещё правила $S_\infty$}
Бесконечная индукция:
$$\infer{(\forall x.\alpha)\vee\delta}{\alpha[x:=\overline{0}]\vee\delta
                                  \quad\alpha[x:=\overline{1}]\vee\delta
                                  \quad\alpha[x:=\overline{2}]\vee\delta\quad\dots}$$

Сечение:
$$\infer{\zeta\vee\delta}{\zeta\vee\alpha\quad\quad\neg\alpha\vee\delta}$$
Здесь $\alpha$ --- секущая формула, число связок в $\neg\alpha$ --- степень сечения.\\
В отличие от других правил, в правиле сечения хотя бы одна из боковых формул $\zeta$ или $\delta$ должна присутствовать.
\end{frame}

\begin{frame}{Дерево доказательства}
\begin{enumerate}
\item Доказательства образуют деревья.
\item Каждой формуле в дереве сопоставим порядковое число (ординал).
\item Порядковое число заключения любого неструктурного правила строго больше порядкового числа его посылок
(больше или равно в случае структурного правила).

%$$%\infer{(\forall a.a = a)_\omega}{
%  % \infer{(0 = 0)_1}{}\quad
%  % \infer{(0'= 0')_2}{\infer{\dots\vphantom{0}}{\infer{(0= 0)_1}{}}}\quad
%  % \infer{0''= 0''}{\infer{\dots\vphantom{0}}{\infer{0'= 0'}{\infer{\dots\vphantom{0}}{\infer{0= 0}{}}}}}\quad\dots
%\infer{(\forall a.a = a)_\omega}{
%   \infer{(0 = 0)_1}{}\quad
%   \infer{(0'= 0')_2}{}\quad
%   \infer{(0''= 0'')_3}{}\quad\dots
%}\quad\quad
%\infer{(\forall a.a = a)_1}{
%   \infer{(0 = 0)_0}{}\quad
%   \infer{(0'= 0')_0}{}\quad
%   \infer{(0''= 0'')_0}{}\quad\dots
%}$$

$$\infer{(\neg\neg\forall x.\neg x'=0)_{\omega+1}}{\infer{(\forall x.\neg x' = 0)_\omega}{(\neg 1=0)_1\quad (\neg 2=0)_2 \quad (\neg 3=0)_4 \quad (\neg 4 = 0)_8 \dots}}$$

\item Существует конечная максимальная степень сечения в дереве (назовём её степенью вывода).
\end{enumerate}
\end{frame}

\begin{frame}{Любая теорема Ф.А. --- теорема $S_\infty$}
\begin{thm}Если $\vdash_\text{фа}\alpha$, то $\vdash_\infty|\alpha|_\infty$ \end{thm}
\begin{exm}Обратное неверно: $$\infer{\forall x.\neg\omega_1(x,\overline{\ulcorner\sigma\urcorner})}
{\neg\omega_1(\overline{0},\overline{\ulcorner\sigma\urcorner})\quad\quad
 \neg\omega_1(\overline{1},\overline{\ulcorner\sigma\urcorner})\quad\quad
 \neg\omega_1(\overline{2},\overline{\ulcorner\sigma\urcorner})\quad\quad\dots}$$
\end{exm}
\begin{thm}Если Ф.А. противоречива, то противоречива и $S_\infty$\end{thm}
\end{frame}

\begin{frame}{Обратимость правил де Моргана, отрицания, бесконечной индукции}
\begin{thm}\vspace{-0.3cm}
$$\infer{\neg\alpha\vee\delta\quad\neg\beta\vee\delta\vphantom{overline{1}]}}{\neg(\alpha\vee\beta)\vee\delta}\quad
  \infer{\alpha\vee\delta\vphantom{overline{1}]}}{\neg\neg\alpha\vee\delta}\quad
  \infer{\alpha[x:=\overline{0}]\vee\delta
          \quad\alpha[x:=\overline{1}]\vee\delta
          \quad\alpha[x:=\overline{2}]\vee\delta\quad\dots}{(\forall x.\alpha)\vee\delta}
$$\vspace{-0.5cm}
%Если формула $\alpha$ доказана и имеет вид, похожий на заключение правил де Моргана, 
%отрицания и бесконечной индукции --- то посылки соответствующих правил могут быть получены из самой 
%формулы $\alpha$ доказательством, причём доказательством с не большей степенью и не большим порядком.
\end{thm}
\begin{proof}\vspace{0.3cm}
\begin{tabular}{ll}\begin{minipage}{0.5\linewidth}
Например, формула вида $\neg\neg \alpha\vee\delta$. \pause

\vspace{0.2cm}Проследим историю $\neg\neg\alpha$; она могла быть получена:
\begin{enumerate}
\item ослаблением --- заменим $\neg\neg\alpha$ на $\alpha$ в этом узле и последующих.
\item отрицанием --- выбросим правило, заменим $\neg\neg\alpha$ на $\alpha$ в последующих.
\end{enumerate}
%Изменённый вывод --- доказательство требуемого.
\end{minipage} &
\begin{minipage}{0.5\linewidth}
\tikz{
  \node at (-1.5,3) (J1) { $\delta(0)$ };
  \node at (1.5,3) (J2) { $\alpha\vee\delta(2)$ };
  \node at (-1.5,1.5) (I1) { ${\color{red}\neg\neg}\alpha\vee\delta(0)$ };
  \node at (1.5,1.5) (I2) { $\color{red}\neg\neg\alpha\vee\delta(2)$ };
  \node at (1.5,0) (C) { ${\color{red}\neg\neg}\alpha\vee\forall x.\delta(x)$ }; 
  \node at (3.5,1.5) (D) { $\dots$ };
  \draw[->] (J1) -- (I1); \draw[->] (I1) -- (C);
  \draw[red,->] (J2) -- (I2); \draw[red,->] (I2) -- (C);
  \draw[->] (D) -- (C);
  \draw[blue,->,bend right=20] (J2) .. controls (0,1.5) .. (C);
}\end{minipage}
\end{tabular}
\end{proof}
\end{frame}

\begin{frame}{Устранение сечений}
\begin{thm}Если $\alpha$ имеет вывод степени $m>0$ порядка $t$, то
можно найти вывод степени строго меньшей $m$ с порядком $2^t$.
\end{thm}

\begin{proof}Трансфинитная индукция. Пусть для всех деревьев порядка $t_1 < t$ 
условие выполнено. Покажем, что оно выполнено для порядка $t$.
Рассмотрим заключительное правило. Это может быть...

\begin{enumerate}
\item Не сечение.
\item Сечение, секущая формула --- элементарная.
\item Сечение, секущая формула --- $\neg\alpha$.
\item Сечение, секущая формула --- $\alpha\vee\beta$.
\item Сечение, секущая формула --- $\forall x.\alpha$.
\end{enumerate}
\end{proof}
\end{frame}

\begin{frame}{Случай 1. Не сечение}
$$\infer{(\alpha)_{t}}{(\pi_0)_{t_0}\quad(\pi_1)_{t_1}\quad(\pi_2)_{t_2}\quad\dots}$$
Заменим доказательства посылок $(\pi_i)_{t_i}$ на $(\pi'_i)_{2^{t_i}}$ по индукционному предположению.

\begin{enumerate}
\item Поскольку степени посылок $m'_i < m_i$, то $\max m'_i < \max m_i$.
\item Поскольку $t_i \le t$, то $2^{t_i} \le 2^t$.
\end{enumerate}
\end{frame}

\begin{frame}{Случай 5. Сечение с формулой вида $\forall x.\alpha$}
\vspace{-0.1cm}$$\infer{\zeta\vee\delta}{\zeta\vee\forall x.\alpha\quad\quad(\neg\forall x.\alpha)\vee\delta}$$\vspace{-0.1cm}
Причём степень и порядок выводов компонент, соответственно, $(m_1,t_1)$ и $(m_2,t_2)$.\vspace{-0.1cm}
\begin{enumerate}
\item По индукции, вывод $\zeta\vee\forall x.\alpha$ можно упростить до $(m_1',2^{t_1})$.
\item По обратимости, можно построить вывод $\zeta\vee\alpha[x := \theta]$ за $(m_1',2^{t_1})$.
\item В формуле $(\neg \forall x. \alpha)\vee\delta$ формула $\neg\forall x.\alpha$ получена
либо ослаблением, либо квантификацией из $\neg\alpha[x := \theta_k]\vee\delta_k$. 
\begin{enumerate}
\item Каждое правило квантификации заменим на:
$$\infer{\zeta\vee\delta_k}{\zeta\vee\alpha[x := \theta_k]\quad\quad(\neg\alpha[x := \theta_k])\vee\delta_k}$$
\item Остальные вхождения $\neg\forall x.\alpha$ заменим на $\zeta$ (в правилах ослабления).
\end{enumerate}
\item В получившемся дереве меньше степень --- так как в $\neg\alpha[x := \theta]$ меньше связок, чем в $\neg\forall x.\alpha$.
%\item Нумерацию можно также перестроить.
\end{enumerate}
\end{frame}

\begin{frame}{Случай 5. Как перестроим доказательство}
\begin{center}\tikz{
    \node (RRW) at (6,5.5) {$\delta_l$};
    \node (RR) at (6,4) {${\color{red}(\neg\forall x.\alpha)}\vee\delta_l$};
    \node (RRnew) at (6.5,3) {$\color{blue}\zeta\vee\delta_l$};
    \node (LL) at (-2,5) {$\dots$};

    \node (R) at (3.2,3) {${\color{red}(\neg\forall x.\alpha)}\vee\delta_k$};
    \node (Rnew) at (1,3) {$\color{blue}\zeta\vee\delta_k$};
    \node (RQ) at (3,5) {$\neg\alpha[x := \theta]\vee\delta_k$};
    \node (L1) at (0,4) {$\color{blue}\zeta\vee\alpha[x := \theta]$};
    \node (L) at (-2,2) {$\color{red}\zeta\vee\forall x.\alpha$};

    \node (CR) at (4.5,2) {${\color{red}(\neg\forall x.\alpha)}\vee\delta$};
    \node (CRnew) at (5,1) {$\color{blue}\zeta\vee\delta$};
    \node (C) at (0,0) {$\color{red}\zeta\vee\delta$};
    \draw[red,->] (L) -- (C);
    \draw[red,->] (LL) -- (L);
    \draw[red,->] (CR) -- (C);
    \draw[blue,->,bend right=30] (LL) to (L1);
    \draw[dashed,->] (R) -- (CR);
    \draw[dashed,->] (RR) -- (CR);
    \draw[double,->,blue] (RR) -- (RRnew);
    \draw[double,->,blue] (CR) -- (CRnew);
    \draw[double,->,blue] (R) -- (Rnew);
    \draw[->] (RRW) -- (RR);
    \draw[->] (RQ) -- (R);
    \draw[blue,->,bend right=20] (RQ) to (Rnew);
    \draw[blue,->] (L1) -- (Rnew);
}\end{center}
\end{frame}

\begin{frame}{Теорема об устранении сечений}
\begin{dfn}Итерационная экспонента
$$(a\uparrow)^m(t
) = 
  \left\{
    \begin{array}{ll}     t,&m=0\\
                          a^{(a\uparrow)^{m-1}(t)},&m > 0
    \end{array}
  \right.
$$
\end{dfn}
\begin{thm}Если $\vdash_\infty\sigma$ степени $m$ порядка $t$, то найдётся доказательство без сечений
порядка $(2\uparrow)^m(t)$
\end{thm}
\begin{proof}
В силу конечности $m$ воспользуемся индукцией по $m$ и теоремой об уменьшении степени.
\end{proof}
\end{frame}

\begin{frame}{Порядок трансфинитной индукции}
\begin{dfn}$\varepsilon_0$ --- неподвижная точка $\varepsilon_0 = \omega^{\varepsilon_0}$\end{dfn}

Иначе говоря, $\varepsilon_0 = \{ \omega, \omega^\omega, \omega^{\omega^\omega}, (\omega \uparrow)^3(\omega), (\omega\uparrow)^4(\omega), \dots \}$.

Очевидно, что теорема об устранении сечений может быть доказана трансфинитной индукцией до ординала $\varepsilon_0$
(максимальный порядок дерева вывода, при правильной нумерации вершин).
\end{frame}

\begin{frame}{Непротиворечивость формальной арифметики}
\begin{thm}Система $S_\infty$ непротиворечива\end{thm}
\begin{proof} \pause
Рассмотрим формулу $\neg 0=0$. 
Если эта формула выводима в $S_\infty$, то она выводима и в $S_\infty$ без сечений.
Тогда какое заключительное правило? \pause
\begin{enumerate}
\item Правило де Моргана? \pause Нет отрицаний дизъюнкции ($\neg(\alpha\vee\beta)\vee\delta$). \pause
\item Отрицание? \pause Нет двойного отрицания ($\neg\neg\alpha\vee\delta$). \pause
\item Бесконечная индукция или квантификация? \pause Нет квантора. \pause
\item Ослабление? \pause Нет дизъюнкции ($\alpha \vee \beta$), в правиле ослабления $\beta$ обязана присутствовать. \pause
\end{enumerate}

То есть, неизбежно, $\neg 0=0$ --- аксиома, что также неверно.
\end{proof}
\end{frame}

\end{document}
