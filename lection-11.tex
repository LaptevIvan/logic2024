\documentclass[aspectratio=169]{beamer}
\usepackage[utf8]{inputenc}
\usepackage[english,russian]{babel}
\usepackage{cancel}
\usepackage{amssymb}
\usepackage{stmaryrd}
\usepackage{cmll}
\usepackage{graphicx}
\usepackage{amsthm}
\usepackage{tikz}
\usepackage{multicol}
\usetikzlibrary{patterns}
\usepackage{chronosys}
\usepackage{proof}
\usepackage{multirow}
\setbeamertemplate{navigation symbols}{}
%\usetheme{Warsaw}

\newtheorem{thm}{Теорема}[section]
\newtheorem{dfn}{Определение}[section]
\newtheorem{lmm}{Лемма}[section]
\newtheorem{exm}{Пример}[section]
\newtheorem{snote}{Пояснение}[section]

\newcommand{\divisible}%
{\mathrel{\lower.2ex%
\vbox{\baselineskip=0.7ex\lineskiplimit=0pt%
\kern6pt \hbox{.}\hbox{.}\hbox{.}}%
}}

\begin{document}

\newcommand\doubleplus{+\kern-1.3ex+\kern0.8ex}
\newcommand\mdoubleplus{\ensuremath{\mathbin{+\mkern-10mu+}}}

\begin{frame}
\begin{center}\LARGE Метод резолюции\end{center}
\end{frame}

\begin{frame}{Метод резолюции (что мы умеем, повторение)}
Дана формула $\alpha$.
\begin{enumerate}
\item Упростим формулу --- поверхностные кванторы всеобщности, сколемизация.

Умеем строить формулу $\beta$:

$$\beta := \forall x_1.\forall x_2.\forall x_k.\delta_1(x_1,\dots,x_k)\with\dots\with\delta_n(x_1,\dots,x_k)$$

$\alpha$ доказуема тогда и только тогда, когда при всех оценках
предикатных и функциональных символов найдётся значение 
сколемовских функций $e_k$, при которых $\beta$ всегда истинна (слоёный пирог из кванторов).

\item Упрощаем предметное множество --- заменили произвольный $D$ на эрбранов универсум $H$. 
Выполнимость формулы эквивалентна выполнимости на эрбрановом универсуме.

\item Осталось избавиться от кванторов всеобщности и организовать правильный перебор
(эрбранов универсум может быть бесконечным).
\end{enumerate}
\end{frame}

\begin{frame}{Оценка формулы на эрбрановом универсуме}
\begin{dfn}
Эрбранов универсум $H_\varphi$ --- всевозможные комбинации функциональных символов из формулы $\varphi$.
Если в формуле нет нульместных функциональных символов, к множеству символов формулы добавляется 
свежий нульместный функциональный символ $a$ и все комбинации с его участием.
\end{dfn}

Например, для $P(0)\vee (P(x)\rightarrow P(x'))$ эрбрановым универсумом будет $\{0, 0', 0'', 0''', \dots\}$,
для $P(x')$ это будет $\{a,a',a'',a''',\dots\}$.

\begin{dfn} Если $\varphi$ --- бескванторная формула, то её эрбранова оценка задаётся как
$\langle H_\varphi, F, P, E\rangle$, функции $F$ определяются как текстовые подстановки
$\llbracket f(\theta)\rrbracket = ``f(`` ++ \llbracket \theta \rrbracket ++ ``)``$, предикаты $P$ задаются перечислением
истинных.
\end{dfn}

Например, для $P(0)\vee (P(x)\rightarrow P(x'))$ эрбранова оценка при истинных предикатах $\{P(0'), P(0''), P(0''''')\}$
такова: $\llbracket \varphi \rrbracket^{ x:=0 } = \text{И}$ и $\llbracket \varphi \rrbracket^{ x:=0'' } = \text{Л}$
\end{frame}

\begin{frame}{Противоречивые системы дизъюнктов}
\begin{thm}[о выполнимости]Формула выполнима тогда и только тогда, когда она выполнима в какой-то эрбрановой оценке.\end{thm}
\begin{proof}Доказано на предыдущей лекции.\end{proof}

\begin{dfn}Система дизъюнктов $S = \{\delta_1,\dots,\delta_n\}$ противоречива,
если для каждой оценки $M = \langle D,P,F,E \rangle$ найдётся $\delta_t$ и такой набор $\overline{d} \in D$,
что $\llbracket\delta_t\rrbracket^{\overline{x} := \overline{d}} = \text{Л}$.
\end{dfn}

\begin{thm}Система дизъюнктов противоречива, если она невыполнима в эрбрановых оценках.\end{thm}
\end{frame}

\begin{frame}{Основные примеры.}

Рассмотрим сколемизированную формулу $\beta$ в КНФ. Заметим, что если $\beta = \forall x_1.\dots.\forall x_k.\delta_1\with\delta_2\with\dots\with\delta_n$,
то $$\vdash \beta \leftrightarrow (\forall x_1.\dots.\forall x_k.\delta_1)\with\dots\with(\forall x_1.\dots.\forall x_k.\delta_n)$$

\begin{dfn}
Дизъюнкт с подставленными значениями из эрбранового универсума $H_\beta$ вместо переменных называется основным примером формулы $\beta$.
\end{dfn}
\begin{exm}Пусть $\beta := \forall x.P(0) \with (P(x)\vee P(x'))$, тогда $P(0''')\vee P(0'''')$ --- основной пример, а $P(0''''')$ --- нет.
\end{exm}

\begin{dfn}Система основных примеров --- все основные примеры, опровергаемые хоть при какой-то эрбрановой оценке $\mathcal{M}$:

\vspace{-0.3cm}
$$\mathcal{E}_S = \{ \delta_t[\overline{x} := \overline{d}]\ |\ \text{существует }\mathcal{M}\text{, что } \llbracket \delta_t[\overline{x} := \overline{d}] \rrbracket_\mathcal{M}=\text{Л};\quad d_i \in H_\beta\}$$
\end{dfn}

\end{frame}

\begin{frame}{Противоречивые множества основных примеров}
\begin{dfn}Система основных примеров $E$ противоречива в эрбрановой оценке (интерпретации), если 
для любой эрбрановой оценки $M$ найдётся такой $\varepsilon\in E$, что $\llbracket \varepsilon \rrbracket_M = \text{Л}$.
\end{dfn}

%\end{dfn}\vspace{-0.5cm}
\begin{thm}Система дизъюнктов $S$ противоречива тогда и только тогда, когда система её всевозможных
основных примеров $\mathcal{E}_S$ противоречива в эрбрановой интерпретации.\end{thm}
%\begin{proof}Для некоторой эрбрановой интерпретации дизъюнкт $\delta_k$
%опровергается тогда и только тогда, когда соответствующая ему подстановка в $\mathcal{E}_S$ опровергается.
%\end{proof}
\end{frame}

\begin{frame}{Теорема Эрбрана}
\begin{thm}[Гёделя о компактности]Если $\Gamma$ --- некоторое семейство бескванторных формул, то $\Gamma$ имеет модель
тогда и только тогда, когда любое его конечное подмножество имеет модель.\end{thm}

\begin{thm}[Эрбрана]Система дизъюнктов $S$ противоречива тогда и только тогда, когда у
$\mathcal{E}_S$ существует конечное противоречивое в эрбрановой интерпретации подмножество.\end{thm}
\begin{proof}$(\Leftarrow)$ 
Пусть $\{\varepsilon_1,\dots,\varepsilon_t\} \subseteq \mathcal{E}_S$ противоречиво, $\varepsilon_i = \delta_{m_i}[\overline{x} := \overline{d_i}]$,
где $\overline{d_i}$ --- набор значений из $H$. 
То есть, для любой эрбрановой оценки $M$ существует $\varepsilon_p$, что $\llbracket\varepsilon_p\rrbracket_M = \text{Л}$. 
%Тогда $\llbracket \delta_{m_p} \rrbracket^{\overline{x}:=\overline{d_p}}_M = \text{Л}$.
Отсюда, по теореме о выполнимости $S$ тоже противоречива.

$(\Rightarrow)$ Если $S$ противоречива, то $\mathcal{E}_S$ противоречива. 
Тогда у неё нет модели. Тогда у неё найдётся конечное противоречивое подмножество (компактность).
\end{proof}

Возможно убедиться в невыполнимости за конечное время.
\end{frame}

\begin{frame}{Общая схема алгоритма}
Цель алгоритма: убедиться, что $\alpha$ доказуемо.
\begin{enumerate}
\item По формуле $\alpha$ строим её отрицание $\neg\alpha$.
\item Приводим к виду с поверхностными кванторами, проводим сколемизацию, находим КНФ:
$\beta = \forall x_1.\dots.\forall x_k.\delta_1\with\dots\with\delta_n$.
\item Убедимся, что при любом $D$ и значениях функциональных и предикатных символов и сколемовских функций $e_k$ найдутся $d_i \in D$, 
что один из дизъюнктов $\delta_t$ при подстановке $\overline{x} := \overline{d}$ ложный.
\item Для этого строим универсум Эрбрана $H$, и систему основных примеров $\mathcal{E}_S$, её противоречивость эквивалентна невыполнимости $\beta$.
\item Конечное противоречивое подмножество обязательно находится в каком-то начальном отрезке $\{\varepsilon_1,\dots,\varepsilon_t\} \subseteq \mathcal{E}_S$ 
(если оно есть).
\end{enumerate}
\end{frame}

\begin{frame}{Пример: как проверяем выполнимость формулы?}
Допустим, формула: $(\forall x.P(x)\with P(x')) \with \exists x.\neg P(x'''')$

\begin{enumerate}
\item Поверхностные кванторы, сколемизация, КНФ: $(\forall x.P(x)) \with (\forall x.P(x')) \with (\neg P(e''''))$
\item Строим эрбранов универсум: $H = \{e, e', e'', e''', \dots \}$
\item Если есть противоречие, то среди основных примеров:
$$\mathcal{E} = \{ P(e), P(e'), P(e''), P(e'''), P(e''''), \neg P(e''''), \dots \}$$
\end{enumerate}

Либо есть $\mathcal{M}$, что $\llbracket\bigwith \mathcal{E}\rrbracket_\mathcal{M} = \text{И}$, 
либо есть $\{\varepsilon_1,\dots,\varepsilon_n\} \subseteq \mathcal{E}$, что $\llbracket\varepsilon_t\rrbracket_\mathcal{M} = \text{Л}$
для какого-то $t$ при каждой эрбрановой оценке $\mathcal{M}$.
\vspace{0.3cm}

\begin{tabular}{lll}
Подмножество $\mathcal{E}$& выполнено в оценке & количество оценок\\\hline
$\{ P(e) \}$ & $\llbracket P(e) \rrbracket  = \text{И}$ & 2 варианта\\
$\{ P(e), P(e') \}$ & $\llbracket P(e) \rrbracket = \llbracket P(e') \rrbracket  = \text{И}$ & 4 варианта\\
\dots\\
$\{ P(e), \dots, P(e''''), \neg P(e'''') \}$ & невыполнимо & 32 варианта
\end{tabular}
\end{frame}

%Добавляем по примеру и проверяем.
%$P(e)$ система выполнима при $\llbracket P(e) \rrbracket  = \text{И}$ (2 варианта)
%$P(e')$ система выполнима при $\llbracket P(e') \rrbracket = \text{И}$ (4 варианта)
%...
%$P(e'''')$ система выполнима при $\llbracket P(e'''') \rrbracket = \text{И}$ (32 варианта)
%$\neg P(e'''')$ система невыполнима (64 варианта).

\begin{frame}{Правило резолюции (исчисление высказываний)}
Пусть даны два дизъюнкта, $\alpha_1 \vee \beta$ и $\alpha_2 \vee \neg \beta$.
Тогда следующее правило вывода называется правилом резолюции:

$$\infer{\alpha_1\vee \alpha_2}{\alpha_1\vee \beta\quad\quad \alpha_2\vee\neg \beta}$$

\begin{thm}Система дизъюнктов противоречива, если в процессе всевозможного применения
правила резолюции будет построено явное противоречие,
т.е. найдено два противоречивых дизъюнкта: $\beta$ и $\neg\beta$.
\end{thm}
\end{frame}

\begin{frame}{Расширение правила резолюции на исчисление предикатов}
Заметим, что правило резолюции для исчисления высказываний не подойдёт для исчисления предикатов.

$$S = \{ P(x), \neg P(0)\}$$

Здесь $P(x)$ противоречит $\neg P(0)$, но правило резолюции для исчисления высказываний здесь неприменимо, потому
что $x$ можно заменять, это не константа:
%$\beta \equiv P(x)$, тогда $\neg \beta \not\equiv \neg P(0)$.

$$\infer{???}{P({\color{red}x})\quad\quad\neg P({\color{red}0})}$$

Нужно заменять $P(x)$ на основные примеры, и искать среди них. Модифицируем правило резолюции для этого.

\end{frame}


\begin{frame}{Алгебраические термы}
	\begin{dfn}Алгебраический терм $$\theta := x\:|\:(f(\theta_1,\ldots,\theta_n))$$ 
где $x-$переменная, $f(\theta_1,\ldots,\theta_n)-$применение функции. Напомним, что константы --- нульместные
функциональные символы, собственно переменные будем обозначать последними буквами латинского алфавита. \end{dfn}
	%\subsection{Уравнение в алгебраических термах $\theta_1=\theta_2$\\Система уравнений в алгебраических термах}
	\begin{dfn}Система уравнений в алгебраических термах
	$
		\begin{cases}
			\theta_1=\sigma_1&\\
			\vdots&\\
			\theta_n=\sigma_n&\\
		\end{cases}
	$\par где $\theta_i \text{ и } \sigma_i-\text{термы}$\par
\end{dfn}
\end{frame}
\begin{frame}{Уравнение в алгебраических термах}
	\begin{dfn}$\{x_i\}=X-$множество переменных, $\{\theta_i\}=T-$множество термов.\end{dfn}
	\begin{dfn}Подстановка$-$отображение вида: $\pi_0:X\to T$, тождественное почти везде (за исключением конечного числа переменных).
        \par $\pi_0(x)$ может быть либо $\pi_0(x)=\theta_i\text{, либо }\pi_0(x)=x$.\end{dfn} 
	Доопределим $\pi:T\to T$, где \begin{enumerate}
		\item $\pi(x)=\pi_0(x)$
		\item $\pi(f(\theta_1, \ldots, \theta_k))=f(\pi(\theta_1), \ldots, \pi(\theta_k))$
	\end{enumerate}
	
	\begin{dfn}Решить уравнение в алгебраических термах$-$найти такую наиболее общую подстановку $\pi$, 
        что $\pi(\theta_1)=\pi(\theta_2)$.
Наиболее общая подстановка --- такая, для которой другие подстановки являются её частными случаями.\end{dfn} 
\end{frame}

\begin{frame}{Задача унификации}
\begin{dfn}
Пусть даны формулы $\alpha$ и $\beta$. Тогда решением задачи унификации
будет такая наиболее общая подстановка $\pi = \mathcal{U}\big[\alpha,\beta\big]$, что $\pi(\alpha) = \pi(\beta)$.

%Будем писать $\Sigma = \mathcal{U}\[\alpha,\beta\]$, если $\Sigma(\alpha) = \Sigma(\beta) = \eta$ и $\Sigma$ --- наиболее общая.
Также, $\eta$ назовём наиболее общим унификатором.
\end{dfn}

\begin{exm}
\begin{itemize}
%\item $\mathcal{U}\[ P(x,g(b)),P(f(a),y) \] = [ x := f(a), y := f(b) ]$ 
%и $P(f(a),g(b))$ --- унификатор.
\item Формулы $P(a,g(b))$ и $P(c,d)$ не имеют унификатора (мы считаем, что $a,b,c,d$ --- нульместные функции, а
$g$ --- одноместная функция).

\item Проверим формулу на соответствие 11 схеме аксиом: $$(\forall x.P(x))\rightarrow P(f(t,g(t),y))$$
Пусть $\pi = \mathcal{U}\big[P(x),P(f(t,g(t),y))\big]$, тогда $\pi(x) = f(t,g(t),y)$.
\end{itemize}
\end{exm}
\end{frame}

\begin{frame}{Правило резолюции для исчисления предикатов}
\begin{dfn}
Пусть $\sigma_1$ и $\sigma_2$ --- подстановки, заменяющие переменные в формуле на свежие. 
Тогда правило резолюции выглядит так:

$$\infer[{\pi = \mathcal{U}\big[\sigma_1(\beta_1),\sigma_2(\beta_2)\big]}]
        {\pi(\sigma_1(\alpha_1)\vee \sigma_2(\alpha_2))}
        {\alpha_1\vee \beta_1\quad\quad\alpha_2\vee\neg \beta_2}$$
\end{dfn}

$\sigma_1$ и $\sigma_2$ разделяют переменные у дизъюнктов, чтобы $\pi$ не осуществила лишние
замены, ведь $\vdash(\forall x.P(x) \with Q(x)) \leftrightarrow (\forall x.P(x))\with(\forall x.Q(x))$, но
$\not\vdash (\forall x.P(x) \vee Q(x)) \rightarrow (\forall x.P(x))\vee(\forall x.Q(x))$.

\begin{exm}\vspace{-0.5cm}
$$\infer[\text{ подстановки: } \sigma_1(x) = x', \sigma_2(x) = x'', \pi(x')=a]{Q(a)\vee T(x'')}{Q(x)\vee P(x) \quad \neg P(a)\vee T(x)}$$
\end{exm}
\end{frame}

\begin{frame}{Метод резолюции}
Ищем $\vdash\alpha$.

\begin{enumerate}
\item будем искать опровержение $\neg\alpha$.
\item перестроим $\neg\alpha$ в КНФ.
\item будем применять правило резолюции, пока получаем новые дизъюнкты и пока 
не найдём явное противоречие (дизъюнкты вида $\beta$ и $\neg\beta$).
\end{enumerate}

Если противоречие нашлось, значит, $\vdash\neg\neg\alpha$. Если нет --- значит, $\vdash\neg\alpha$.
Процесс может не закончиться.
\end{frame}

\begin{frame}{SMT-решатели}

Обычно требуется не логическое исчисление само по себе, а теория первого порядка.
То есть, <<Satisfability Modulo Theory>>, <<выполнимость в теории>> --- вместо SAT, выполнимости.
\begin{itemize}
\item Иногда можно вложить теорию в логическое исчисление, 
даже в исчисление высказываний: $\overline{S_2S_1S_0} = \overline{A_1A_0}+\overline{B_1B_0}$
$$\begin{array}{ll}
S_0 = A_0 \oplus B_0 & C_0 = A_0 \with B_0\\
S_1 = A_1 \oplus B_1 \oplus C_0 & C_1 = (A_1 \with B_1) \vee (A_1 \with C_0) \vee (B_1 \with C_0) \\
S_2 = C_1\end{array}$$

\item А можно что-то добавить прямо на уровень унификации / резолюции:
Например, можем зафиксировать арифметические функции --- и производить вычисления
в правиле резолюции вместе с унификацией.

Тогда противоречие в $\{x = 1+3+1,\neg x = 5\}$ можно найти за один шаг.
\end{itemize}
\end{frame}

\begin{frame}[fragile]{Уточнённые типы (Refinement types), LiquidHaskell}
\begin{dfn}(Неформальное) Уточнённый тип --- тип вида $\{\tau(x)\ |\ P(x)\}$, где $P$ --- некоторый предикат.\end{dfn}

Пример на LiquidHaskell:
\begin{verbatim}
data [a] <p :: a -> a -> Prop> where
   | []  :: [a] <p>
   | (:) :: h:a -> [a<p h>]<p> -> [a]<p>
\end{verbatim}
\begin{itemize}
\item \verb!h:a! --- голова ($h$) имеет тип $a$\\
\item \verb![a<p h>]<p>! --- хвост состоит из значений типа $a$, уточнённых $p$ --- $\{ t : a\ |\ p\ h\ t\}$ (карринг: \verb!a <p h>!).
\end{itemize}

\begin{verbatim}
{-@ type IncrList a = [a] <{\xi xj -> xi <= xj}> @-}
{-@ insertSort    :: (Ord a) => xs:[a] -> (IncrList a) @-}
insertSort []     = []
insertSort (x:xs) = insert x (insertSort xs) 
\end{verbatim}
\end{frame}

\end{document}
